\begin{frame}
      \frametitle{Conclusions}
      \begin{itemize}
        \item Transition analysis and sensitivity analysis are methods
              to investigate the potential needs of a nuclear fuel cycle
        \item Demonstrated a methodology to investigate a fuel cycle transition
        \item<2-> The reactors deployed influence the materials needed by 
              the fuel cycle
        \item<2-> The deployment scheme strongly affects the material requirements
        \item<3-> Tradeoff between \gls{HALEU} requirements and \gls{SNF} 
              discharged between the Xe-100 and VOYGR
        \item<3-> \glspl{MMR} require more \gls{SWU} capacity, but less fuel mass
              compared with VOYGRs          
        \item<4-> All of the sensitivity analysis scenarios require a cumulative 
              \gls{HALEU} mass between 2,710 - 37,500 MT between 2025-2090       
      \end{itemize}
\end{frame}

\begin{frame}
  \frametitle{Future Work}
  \begin{block}{Ongoing Work}
    
  \begin{itemize}
    \item Consider recycling scenarios (close the fuel cycle)
    \begin{itemize}
      \item Couple \Cyclus with OpenMC's depletion solver
    \end{itemize}
    \item Perform optimization of these fuel cycles to minimize material
          requirements.
    \item Investigate some of the effects of using downblended uranium 
          in advanced reactors
  \end{itemize}
  \end{block}

  \pause
  \begin{block}{Future Work}
    \begin{itemize}
      \item Convert these mass/capacity needs to facility sizes and numbers
      \item Investigate costs for developing facilities
      \item Apply limits on uranium sources and facility throughputs
      \item<1-> Investigate safeguards for advanced reactor fuel cycles
      \item<1-> Research fast reactors, their fuel cycle, and safeguards for them
    \end{itemize}
  \end{block}
\end{frame}