\documentclass[9pt,xcolor={table}]{beamer}
\setbeamertemplate{caption}[numbered]
\usetheme[white]{Illinois}
%\title[short title]{long title}
\title[Title]{Fuel cycle needs of deploying advanced reactors}
\subtitle{UTK NE Department Colloquium}
\author[Amanda M. Bachmann]{Amanda M. Bachmann\\ Advanced Reactors and Fuel Cycles Group}
\date[04.19.2023]{April 19, 2023}
\institute[UIUC]{University of Illinois at Urbana-Champaign}


%\usepackage[table]{xcolor}
\usepackage{amsfonts}
\usepackage{amsmath}
\usepackage{xspace}
\usepackage{graphicx}
\usepackage{caption}
\usepackage{subcaption}
\usepackage{booktabs} % nice rules for tables
\usepackage{microtype} % if using PDF
\usepackage{bigints}

\graphicspath{{images/}}

\newcommand{\units}[1] {\:\text{#1}}%
\newcommand{\SN}{S$_N$}%{S$_\text{N}$}%{$S_N$}%
\newcommand{\Cyclus}{\textsc{Cyclus}\xspace} %
\newcommand{\Cycamore}{\textsc{Cycamore}\xspace} %
\DeclareMathOperator{\erf}{erf}
%I need some complimentary error functions... 
\DeclareMathOperator{\erfc}{erfc}
%Those icons in the references are terrible looking
\setbeamertemplate{bibliography item}[text]

%%%% Acronym support

\usepackage[acronym,toc]{glossaries}
\input{acros}

\usepackage{tikz}
\usetikzlibrary{shapes.geometric, arrows, backgrounds}
\usetikzlibrary{positioning, arrows, decorations, shapes, matrix, fit, tikzmark}

\tikzstyle{agent} = [rectangle, rounded corners, minimum width=0.1cm, minimum height=0.2cm,text centered, draw=black, fill=blue!30]
\tikzstyle{transition} = [rectangle, rounded corners, minimum width=0.1cm, minimum height=0.2cm,text centered, draw=black, fill=red!30]
\tikzstyle{arrow} = [thick,->,>=stealth]

\tikzstyle{region} = [rectangle, rounded corners, minimum width=0.1cm, minimum height=0.2cm,text centered, draw=black, fill=green!30]
\tikzstyle{institution} = [rectangle, rounded corners, minimum width=0.1cm, minimum height=0.2cm,text centered, draw=black, fill=red!30]
\tikzstyle{facility} = [rectangle, rounded corners, minimum width=0.1cm, minimum height=0.2cm,text centered, draw=black, fill=blue!30]
\tikzstyle{connect} = [thick,-]

\usepackage[beamer,customcolors]{hf-tikz}
\tikzset{hl/.style={
    set fill color=yellow!80!black!40,
    set border color=yellow!80!black,
  },
}

\def\firstcircle{(0,0) circle (2cm)}
\def\secondcircle{(60:3cm) circle (2cm)}
\def\thirdcircle{(0:3cm) circle (2cm)}
\makeglossaries

%try to get rid of header on title page\dots
\makeatletter
    \newenvironment{withoutheadline}{
        \setbeamertemplate{headline}[default]
        \def\beamer@entrycode{\vspace*{-\headheight}}
    }{}
\makeatother

\makeatother
\setbeamertemplate{footline}
{
  \leavevmode%
  \hbox{%
    \rightline{\insertframenumber{} / \inserttotalframenumber\hspace*{1ex}}
  }%
  \vskip0pt%
}
\makeatletter
\begin{document}
%%%%%%%%%%%%%%%%%%%%%%%%%%%%%%%%%%%%%%%%%%%%%%%%%%%%%%%%%%%%%
%% From uw-beamer Here's a handy bit of code to place at 
%% the beginning of your presentation (after \begin{document}):
\newcommand*{\alphabet}{ABCDEFGHIJKLMNOPQRSTUVWXYZabcdefghijklmnopqrstuvwxyz}
\newlength{\highlightheight}
\newlength{\highlightdepth}
\newlength{\highlightmargin}
\setlength{\highlightmargin}{2pt}
\settoheight{\highlightheight}{\alphabet}
\settodepth{\highlightdepth}{\alphabet}
\addtolength{\highlightheight}{\highlightmargin}
\addtolength{\highlightdepth}{\highlightmargin}
\addtolength{\highlightheight}{\highlightdepth}
\newcommand*{\Highlight}{\rlap{\textcolor{HighlightBackground}{\rule[-\highlightdepth]{\linewidth}{\highlightheight}}}}
\colorlet{lightblue}{blue!40!}
\definecolor{lightorange}{HTML}{FAA21A}
\colorlet{lightpink}{red!20!}

\tikzset{   
        every picture/.style={remember picture,baseline},
        every node/.style={anchor=base,align=center,outer sep=1.5pt},
        every path/.style={thick},
        }

\newcommand\marktopleft[1]{%
    \tikz[overlay,remember picture] 
        \node (marker-#1-a) at (.1em,.3em) {};%
}
\newcommand\markbottomright[1]{%
    \tikz[overlay,remember picture] 
        \node (marker-#1-b) at (.1em,.3em) {};%
    \tikz[overlay,remember picture,inner sep=3pt]
        \node[draw=red,rounded corners,fit=(marker-#1-a.north west) (marker-#1-b.south east)] {};%
}

%%%%%%%%%%%%%%%%%%%%%%%%%%%%%%%%%%%%%%%%%%%%%%%%%%%%%%%%%%%%%
%%--------------------------------%%
\begin{withoutheadline}
    \frame{
      \titlepage
    }
    \end{withoutheadline}

%%--------------------------------%%
\AtBeginSection[]{
\begin{frame}
  \frametitle{Outline}
  \tableofcontents[currentsection]
\end{frame}
}

\section{About Me}
\begin{frame}
    \frametitle{My background}
    Education 
    \begin{itemize}
        \item BS in Nuclear Engineering, University of Tennessee, Knoxville (2019)
        \item MS in Nuclear Engineering, University of Tennessee, Knoxville (2020)
        \item PhD in NPRE, University of Illinois Urbana-Champaign (In Progress, est. 2023)
    \end{itemize}
    Research Experience
    \begin{itemize}
        \item Multivariate modeling of radiation signatures for safeguards, using ORIGEN
        \item Modeling material flow through a pyroprocessing facility, using ORIGAMI
        \item Comparing effects of Doppler broadening methods in SHIFT (ORNL)
        \item Investigating fuel cycle impacts of using \gls{HALEU} in reactors,
              using Cyclus, Dakota, and OpenMC (UIUC/ANL)
    \end{itemize}

\end{frame}

\begin{frame}
    \frametitle{Research Interests}
    \begin{figure}[t]
    \centering
    \begin{tikzpicture}[node distance=0.6cm]
        \draw \firstcircle node[below] {};
        \draw \secondcircle node [above] {};
        \draw \thirdcircle node [below] {};
    
        % Next, we want the highlight the intersection of all three circles:
    
        %\begin{scope}
        %  \clip \firstcircle;
        %  \clip \secondcircle;
        %  \fill[green!30] \thirdcircle;
        %\end{scope}
        
        \node[align=center] at (3.5,-0.5) {Scientific\\Computing};
        \node[align=center] at (-0.5,-0.5) {Nonproliferation\\Safeguards};
        \node[align=center] at (1.5,2.75) {Nuclear\\Fuel Cycle};
        \node[align=center] at (1.5,0.84) {Me};
    \end{tikzpicture}
\end{figure}
\end{frame}
\section{Introduction}
\input{intro}
\section{Transition analysis}
\input{transition_analysis}
\subsection{Results}
\input{results}
\section{Sensitivity analysis}
\subsection{Methodology}
\begin{frame}
    \frametitle{Transition analysis doesn't tell the full story}
    \begin{itemize}
        \item From the transition analysis we learned how changing which 
              reactor is predominantly built affects the material 
              requirements
        \item There are so many more parameters and assumptions in the scenario modeled
        \item We can use sensitivity analysis to see the effect of other model 
              parameters on the material requirements
    \end{itemize}
\end{frame}

\begin{frame}
    \frametitle{Sensitivity analysis provides insight into how decisions affect material needs}
    \begin{itemize}
        \item Vary a single parameter at a time and observe the output
        \item Helps to identify different relationships about the transition
        \item Couple \Cyclus with Dakota \cite{adams_dakota_2019}, model 
              variations in Scenario 7
    \end{itemize}
    \begin{columns}
        \column[t]{5cm}<2->
        Model inputs (parameters)
        \begin{itemize}
            \item Transition start time, January 2025-January 2040 
            \item LWR lifetime: percent of LWR fleet operating 
                  for 80 years, 0-50\%
            \item Build share of each advanced reactor, 0-50\%
            \item \gls{MMR} \& Xe-100 discharge burnup
        \end{itemize}

        \column[t]{5cm}<3->
        Model outputs (metrics, material requirements)
        \begin{itemize}
            \item Total fuel mass 
            \item HALEU mass
            \item Total \gls{SWU}
            \item SWU to produce HALEU
            \item Feed to produce HALEU
            \item \gls{SNF} mass
        \end{itemize}
    \end{columns}
\end{frame}

\begin{frame}
    \frametitle{Adjust deployment when varying the build share}
    \begin{columns}
        \column{4.5cm}
            \begin{itemize}
                \item Deploy the specified reactor first to meet the build share
                \item Deploy the other reactors using the previous scheme (largest 
                      to smallest power output) to meet the remaining demand
                \item Deployment schedule is given to \Cyclus
            \end{itemize}

        \column{6cm}
            \begin{figure}
                \centering 
                \includegraphics[scale=0.33, trim=100 100 50 50,clip]{VOYGR_build_share.pdf}
                \caption{Deployment of advanced reactors to meet 
                a demand of 530 MWe and a 50\% VOYGR build share.}
                \label{fig:build_share_deployment}
            \end{figure}
    \end{columns}
\end{frame}
\subsection{Results}
\begin{frame}
    \frametitle{Varying input: MMR build share}
    \begin{columns}
        \column{4.5cm}
            \begin{itemize}
                \item All of the metrics increase as build share increases
                \item \gls{SNF} mass has the smallest increase in 
                      relative change
                \item \gls{HALEU} \gls{SWU} capacity has the largest relative increase 
                \item These results are primarily because of the deployment scheme    
            \end{itemize}
        \column{5.5cm}
            \begin{figure}
                \includegraphics[scale=0.45]{mmr.pdf}
                \caption{Relative change in metrics as a function MMR 
                build share.}
                \label{fig:mmr}
            \end{figure}

\end{columns}
\end{frame}

\begin{frame}
    \frametitle{Varying MMR build share -- Number of reactors deployed}
    \begin{figure}
        \centering
        \includegraphics[scale=0.6]{mmr_combined_reactors.pdf}
        \caption{Number of MMRs deployed as a function of time when the 
        MMR build share is varied.}
    \end{figure}
\end{frame}
\begin{frame}
    \frametitle{Varying input: Xe-100 build share}
    \begin{columns}
        \column{4.5cm}
            \begin{itemize}
                \item The \gls{HALEU}-related metrics increase
                      with increasing build share
                \item Total \gls{SNF} and total fuel mass decrease with 
                      increasing build share
                \item Total \gls{SWU} capacity is relatively constant, at most 
                      a relative change of 1.06
            \end{itemize}
        \column{5.5cm}
            \begin{figure}
                \includegraphics[scale=0.45]{xe100.pdf}
                \caption{Relative change in metrics as a function of Xe-100 
                build share \protect\cite{bachmann_sensitivity_2022}.}
                \label{fig:xe100}
            \end{figure}

\end{columns}
\end{frame}
   

\begin{frame}
    \frametitle{Varying input: VOYGR build share}
    \begin{columns}
        \column{4.5cm}
            \begin{itemize}
                \item Fuel mass and SNF mass increase while \gls{HALEU}
                      metrics decrease 
                \item Total SWU capacity is relatively constant
                \item Opposite trends as increasing Xe-100 build share
                \item As the build share increases, the number of Xe-100s
                      decreases and the number of MMRs is fairly constant
                
            \end{itemize}
        \column{5.5cm}
        \begin{figure}
            \centering 
            \includegraphics[scale=0.45]{voygr.pdf}
            \caption{Relative change in metrics as a function 
            of VOYGR build share \protect\cite{bachmann_sensitivity_2022}.}
            \label{fig:voygr}
        \end{figure}
    \end{columns}
\end{frame}
\section{Conclusions}
\begin{frame}
      \frametitle{Conclusions}
      \begin{itemize}
        \item Transition analysis and sensitivity analysis are methods
              to investigate the potential needs of a nuclear fuel cycle
        \item Demonstrated a methodology to investigate a fuel cycle transition
        \item<2-> The reactors deployed influence the materials needed by 
              the fuel cycle
        \item<2-> The deployment scheme strongly affects the material requirements
        \item<3-> Tradeoff between \gls{HALEU} requirements and \gls{SNF} 
              discharged between the Xe-100 and VOYGR
        \item<3-> \glspl{MMR} require more \gls{SWU} capacity, but less fuel mass
              compared with VOYGRs          
        \item<4-> All of the sensitivity analysis scenarios require a cumulative 
              \gls{HALEU} mass between 2,710 - 37,500 MT between 2025-2090       
      \end{itemize}
\end{frame}

\begin{frame}
  \frametitle{Future Work}
  \begin{block}{Ongoing Work}
    
  \begin{itemize}
    \item Consider recycling scenarios (close the fuel cycle)
    \begin{itemize}
      \item Couple \Cyclus with OpenMC's depletion solver
    \end{itemize}
    \item Perform optimization of these fuel cycles to minimize material
          requirements.
    \item Investigate some of the effects of using downblended uranium 
          in advanced reactors
  \end{itemize}
  \end{block}

  \pause
  \begin{block}{Future Work}
    \begin{itemize}
      \item Convert these mass/capacity needs to facility sizes and numbers
      \item Investigate costs for developing facilities
      \item Apply limits on uranium sources and facility throughputs
      \item<1-> Investigate safeguards for advanced reactor fuel cycles
      \item<1-> Research fast reactors, their fuel cycle, and safeguards for them
    \end{itemize}
  \end{block}
\end{frame}
\begin{frame}
    \frametitle{Acknowledgements}
    This material is based upon work supported under a University 
        Nuclear Leadership Program Graduate Fellowship. Any opinions, findings, conclusions, or 
    recommendations expressed in this publication are those of the author(s) 
    and do not necessarily reflect the views of the Department of Energy Office 
    of Nuclear Energy.
        
    I would also like to thank:
    \begin{itemize}
        \item Madicken Munk
        \item Scott Richards
        \item Bo Feng
        \item TK Kim
        \item RFCA/ARFC Group members
    \end{itemize}

\end{frame}


\begin{frame}[allowframebreaks]
    \frametitle{References}
    \bibliographystyle{plain}
    {\footnotesize \bibliography{bibliography} }
  
\end{frame}

\begin{frame}
  \frametitle{Questions?}
    \begin{figure}
      \centering
      \includegraphics[scale=0.3]{littleR.jpg}
    \end{figure}
\end{frame}

\begin{frame}
    \frametitle{Enriched uranium masses}
    \begin{table}
        \centering 
        \caption{Metrics for enriched uranium sent to advanced 
        reactors between 2025-2090 in Scenarios 2-7.}
        \label{tab:nogrowth_uranium}
        \begin{tabular}{l p{2cm} p{2cm} p{2cm} p{2cm}}
            \hline
            Scenario & Average (MT/month) & \gls{HALEU} Average 
            (MT/month) & Maximum (MT)& Cumulative (MT)\\\hline
            2 & 88.90 & 88.90 & 1,007 & 69,251\\
            3 & 31.64 & 31.64 & 86.79 & 24,646\\
            4 & 33.62 & 33.62 & 101.5 & 26,193\\
            5 & 152.3 & 3.070 & 581.1 & 118,608\\
            6 & 39.92 & 29.51 & 103.3 & 31,095\\
            7 & 33.85 & 32.97 & 103.0 & 26,370\\
            \hline
        \end{tabular}
    \end{table}
\end{frame}

\begin{frame}
    \frametitle{SWU capacity}
    \begin{table}
        \centering 
        \caption{Metrics for \gls{SWU} capacity to enrich uranium for 
        advanced reactors between 2025-2090 in Scenarios 2-7.}
        \label{tab:nogrowth_swu}
        \begin{tabular}{l p{2cm} p{2cm} p{2cm}}
            \hline
            Scenario & Average  (MT-SWU/month) & Average
            for \gls{HALEU} (MT-SWU/month) & Maximum (MT-SWU)\\\hline
            2 & 4,010 & 4,010 & 45,424 \\
            3 & 1,090 & 1,090 & 2,991\\
            4 & 1,192 & 1,192 & 3,668\\
            5 & 1,145 & 138.5 & 4,228 \\
            6 & 1,087 & 1,017 & 3,050\\
            7 & 1,167 & 1,161 & 3,735\\
            \hline
        \end{tabular}
    \end{table}
\end{frame}

\begin{frame}
    \frametitle{SNF discharged}
    \begin{table}
        \centering 
        \caption{Metrics for waste discharged from advanced reactors 
        between 2025-2090 in Scenarios 2-7.}
        \label{tab:nogrowth_waste}
        \begin{tabular}{l p{2cm} p{2cm} p{2cm} p{2cm}}
            \hline
            Scenario & Average (MT/month) & Average of \gls{HALEU} 
            (MT/month) & Maximum (MT) & Cumulative (MT)\\\hline
            2 & 66.15 & 66.15 & 1,142 & 51,531\\
            3 & 32.93 & 32.93 & 55.29 & 25,654\\
            4 & 34.07 & 34.07 & 77.55 & 26,543\\
            5 & 144.9 & 2.294 & 377.0 & 112,913\\
            6 & 40.68 & 30.72 & 75.12 & 31,691\\
            7 & 34.46 & 33.62 & 83.14 & 26,848\\
            \hline
        \end{tabular}
    \end{table}

\end{frame}

\begin{frame}
    \frametitle{Transition Start Time}
    \begin{figure}
        \includegraphics[scale=0.5]{ts.pdf}
        \caption{Change in metrics from varying the transition start time.}
        \label{fig:ts}
    \end{figure}
\end{frame}

\begin{frame}
    \frametitle{LWR Lifetime}
    \begin{figure}
        \includegraphics[scale=0.5]{lwr.pdf}
        \caption{Change in metrics from varying the percent of 
        the LWR fleet operating for 80 years.}
        \label{fig:lwr}
    \end{figure}
\end{frame}

\begin{frame}
    \frametitle{MMR Burnup}
    \begin{figure}
        \includegraphics[scale=0.5]{mmr_bu.pdf}
        \caption{Change in metrics from varying the MMR burnup.}
        \label{fig:mmr_bu}
    \end{figure}
\end{frame}


\begin{frame}
    \frametitle{Xe-100 burnup}
    \begin{figure}
        \includegraphics[scale=0.5]{xe100_bu.pdf}
        \caption{Change in metrics from varying the Xe-100 burnup.}
        \label{fig:xe100_bu}
    \end{figure}
\end{frame}


\end{document}
