\subsection{Motivation}
\begin{frame}
    \frametitle{The US is looking to develop supplies of HALEU}
    \begin{columns}
        \column[t]{5cm}
    \begin{itemize}
    \item Multiple new reactor designs require \gls{HALEU} fuel, which allows for: 
    \begin{itemize}
        \item Longer cycle time
        \item Increased capacity factor
        \item Higher burnup 
    \end{itemize}
    \item<2-> The US does not have any commercial supplies of \gls{HALEU}
    \item<3-> To meet the \gls{HALEU} demand, the U.S. \gls{DOE} has proposed two methods \cite{griffith_overview_2020}:
    \begin{itemize}
        \item Enrichment of natural uranium
        \item Recovery and downblending of \gls{HEU}
    \end{itemize}
    
    \end{itemize}

    \column[t]{5cm}
    \begin{table}
        \centering
        \caption{Categories of uranium enrichment by weight fraction of 
        $^{235}$U.}
        \label{tab:enrichemnt}
        \begin{tabular}{l c c}
            \hline
            Category & Weight fraction (\%)\\\hline
            Depleted & $<$0.711 \\
            Natural & 0.711 \\
            LEU & 0.711-20 \\
            \gls{HALEU} & 5-20 \\
            \gls{HEU} & $\ge$20 \\
            \hline
        \end{tabular}
    \end{table}
    \end{columns}
\end{frame}

\subsection{Background}
\begin{frame}
    \frametitle{Nuclear fuel cycle}
    \begin{figure}[t]
    \centering
    \begin{tikzpicture}[node distance=1cm]
        \node (mine) [agent] {Mine};
        \node (mill) [agent, right of=mine] {Mill};
        \node (conversion) [agent, right of=mill, xshift=0.5cm] {Conversion};
        \node (enrichment) [agent, right of=conversion, xshift=1cm]{Enrichment};
        \node (fabrication) [agent, right of=enrichment, xshift=1.5cm]{Fuel Fabrication};
        \node (reactor) [agent, below of=fabrication, yshift=-1.5cm]{Reactor};
        \node (storage) [agent,  left of=reactor, xshift=-1cm]{Storage};
        \node (sinkhlw) [agent, left of=storage, xshift=-1cm]{Repository};


        \draw [arrow] (mine) --  (mill); 
        \draw [arrow] (mill) -- (conversion); 
        \draw [arrow] (conversion) -- (enrichment);
        \draw [arrow] (enrichment) -- (fabrication);
        \draw [arrow] (fabrication) -- (reactor);
        \draw [arrow] (reactor) -- (storage);
        \draw [arrow] (storage) -- (sinkhlw);
        \pause
        \node (reprocessing) [transition, above of=storage]{Reprocessing};
        
        \draw [arrow] (storage) -- (reprocessing);
        \draw [arrow] (reprocessing) -- (fabrication);
        \draw [arrow] (reprocessing) -- (sinkhlw);
        \pause
        \node (downblending) [region, above of=fabrication]{Downblending};
        \draw [arrow] (downblending) -- (fabrication);

        \end{tikzpicture}
    \caption{Overview of the Nuclear Fuel Cycle.}
    \label{fig:fuel_cycle}
\end{figure}
\end{frame}

\begin{frame}
    \frametitle{Uranium enrichment}
    \begin{itemize}
        \item Process to increase the relative abundance of specific
              isotopes of an element
        \pause
        \item The throughput of a facility is based on the product 
              mass, product assay, and the \gls{SWU} capacity
    \end{itemize}
    \pause
    \vspace{-0.2cm}
    \begin{columns}
        \column{6.5cm}
            \begin{align*}
                    & F = P + T \\
                    & x_fF = x_pP + x_tT\\
                    & SWU = \left[P\times V(x_p) +T\times V(x_t) - F\times V(x_f)\right]*t\\
                    & \text{in which:}\\
                    & V(x_i) = (2x_i - 1)*\ln\left(\frac{x_i}{1-x_i}\right)
            \end{align*}
            \vspace{-0.5cm}
            \begin{figure}[t]
    \centering
    \begin{tikzpicture}[node distance=1cm]
        \node (conversion) [agent] {\small Conversion};
        \node (enrichment) [agent, right of=conversion, xshift=1.25cm]{\small Enrichment};
        \node (fabrication) [agent, right of=enrichment, xshift=1.25cm]{\small Fuel Fabrication};
        \node (sink) [agent, below of=enrichment]{\small Disposal};
        
        \draw [arrow] (conversion) -- node[anchor=south]{F} (enrichment);
        \draw [arrow] (enrichment) -- node[anchor=south]{P}(fabrication);
        \draw [arrow] (enrichment) -- node[anchor=east]{T}(sink);

        \end{tikzpicture}
\end{figure}
    \column{3.5cm}
    \begin{table}
        \centering
        \vspace{-0.3cm}
        \begin{tabular}{c m{2cm}}
            \hline
            Variable & Definition \\
            \hline
            F & feed mass \\
            P & product mass \\
            T & tails mass\\
            x$_i$ & assay of material stream \\
            SWU & Separative Work Units\\
            V(x$_i$) & separation potential function\\
            \hline
        \end{tabular}
    \end{table}

    \end{columns}
\end{frame}

\begin{frame}
    \frametitle{Efforts to estimate HALEU needs}
    Efforts are underway to estimate potential \gls{HALEU} needs:
    \begin{itemize}
        \item \gls{NEI} surveyed multiple reactor design companies
              to estimate \gls{HALEU} needs between now and 2035 
              \cite{korsnick_updated_2021,nuclear_energy_institute_establishing_2022}
        \item \gls{DOE} labs modeled the transition to some 
              \gls{HALEU}-fueled reactors to estimate \gls{HALEU} needs 
              to meet current net-zero carbon goals in 2050 \cite{dixon_estimated_2022}
    \end{itemize}
    \pause
    This previous work is all based on announced advanced reactor projects.
    \begin{itemize}
        \item What happens if we scale up on reactor deployment really quickly?
        \item How does decision making affect the fuel cycle requirements?
    \end{itemize}
\end{frame}

\subsection{Objectives}
\begin{frame}
    \frametitle{Objectives}
        \begin{itemize}
        \item Explore how the deployment of \gls{HALEU}-fueled reactors 
              affects the US nuclear fuel cycle
        \item Quantify potential material requirements for the transition from 
              \glspl{LWR} to advanced reactors in a once-through and recycling 
              fuel cycle
        \item Understand the impacts of fuel cycle parameters on the 
              material requirements and design optimized transition scenarios
        \item Identify potential limitations in using downblended \gls{HEU} 
              in advanced reactors
        \end{itemize}
\end{frame}